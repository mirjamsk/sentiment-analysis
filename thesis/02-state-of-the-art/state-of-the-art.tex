This chapter describes state of the art of sentiment analysis in social media.
Chapter consists of three sections, each of them trying to bring closer the need of sentiment analysis in current market:

\begin{enumerate}
	\item Need for sentiment analysis
	\item Application of sentiment analysis in various companies and non-profit organizations
	\item Most used tools for sentiment analysis
\end{enumerate}

\section{Need for sentiment analysis}
With growth of people's interaction and company's advertisements through social media, we have come to the point of realizing that people sharing opinions could help us "predict" stock market and as well follow current trends by guiding the market according to the customers input.
Customers nowadays have endless ways to interact with brands which could help increasing brand's awareness but if not properly analyzed could also lead to obtaining not quite accurate view of customer's satisfaction.
The idea of analyzing customer opinion has driven companies to search for an automated way of understanding what message are customers sharing online. The main network of spreading opinions is social media. Almost every tweet, comment, re-share or  review gives an information that could guide a company towards better planning , optimizing production and better stock managing.
Reason for finding an automated way of analyzing customer's opinion comes from a problem of big data being generated each day which makes impractical of doing human analysis of each user input. Leaving the big data problem aside, brings us to another issue; being able to beat natural language processing challenge. Reason for making the task harder is that user input might be informal, "slang like content with emojis , hash tags, even full with sarcastic sentences which would lead to unreliable results of sentiment analysis.

\section{Application of sentiment analysis in various companies and non-profit organizations}

\section{Most used tools for sentiment analysis}

\subsection*{Commercial solutions}
As every commercial product, basic goal is user satisfaction. Commercial solutions provide user with rich customizable, easy to use interfaces for a not so fair price. By paying for the service users, usually medium to large scale companies, receive a platform which contains algorithms for data analysis used as a black box and detailed colorful visualization tools for representing results of the analysis. One of important issues that users wouldn’t deal with, as they would if building their own solutions, is that such platforms usually come with needed infrastructure to support such data intense analysis. Here we will mention few most widely used commercial tools.\\
\textit{Google Analytics\\}
Google Analytics helps you know your audience, find your best content, and optimize ad inventory. Providing you with real-time reports of what is happening on your site right now so you can make adjustments fast. Engagement metrics help you see what is working, while integrations with Google and publisher tools like AdSense, DoubleClick AdExchange, and DoubleClick for Publishers (Analytics 360 only) make it easy to package and sell your ad inventory.  Google has developed a solution which enables the user to gather data, preprocess it, and train a model using Google Prediction API like a black box.\\
\textit{Sales Force Marketing Cloud solution - Radian6\\}
Most certainly that human sentiment analysis is the most accurate method even if you think how much human differ in their interpretation. Radian6 has introduced an automated sentiment analysis tool which has flexibility to allow users to change the perspective of analysis. If you do real sentiment evaluation manually, you will obtain more accurate results than any other automated tool could give you. Given a simple example, if a user compares different beverage brands, most likely he would rate better the beverage he prefers based on the prevailing taste of it. Radian6 solution will enable the user to do deeper analysis into specific topics via different types of ad hoc analysis
Radian6 has given various solutions to fill the gap between marketing and customer satisfaction by using social insights to drive marketing campaigns. By listening, engaging and analyzing data on social media, users are able to create sales plans which could lead to better stock planning.\\
\textit{Brandwatch\\}
Brandwatch Analytics is a web-based platform with monthly subscription basis with different range of packages meeting needs of various scaled companies. They search and store date based on users query on the market. Quite accurately guarantees spam free and duplicate free data. With the gathered data they assure you of optimizing marketing in social media. The platform offers various customizations that could accommodate to the needs of the user. By acquiring data every day and providing users with tools to analyze and visualize them, they have convinced a lot of famous brands that Brandwatch is a good tool to help them make data-driven market decisions such as Cisco, British Airways and Dell. 
Good thing about Brandwatch as a commercial solution is that it provides coverage of various data sources, independent of language barrier or data quality. Besides of the coverage advantage, it provides stable analytic tools, as well as visualization tools.  It is mostly used by large companies that could afford the platform.\\
\subsection*{Open source solutions}
Main benefits of adopting an open source solution are lower costs, in this case using an open source library is free, as well as trend of keeping an open source solution always available because it is usually maintained by a community. For a commercial solution, it could happen that a vendor shuts down his business and with it taking its software out of market. Another major advantages of using an open source solution is that often there is a collaboration between libraries and as well as variety of available solutions. Open source solutions are not bind to changes and updates to releases; instead they can be developed collaboratively when functionality is needed. Of course using an open source library can have its down sides, a library or an API call can be limited by number of usages by day or by accepted amount of data it can receive. Thus implicates that these kind of solutions are not setting up an infrastructure that could handle data-intensive analysis and are usually used for educational purposes.\\
\textit{Natural Language Toolkit\\}
Natural Language Toolkit is an open source platform for building programs which work with textual data. It is equipped with various libraries for text processing which provide tokenization, tagging, stemming and handy wrappers around NLP libraries. NLTK has a very detailed documentation which can guide a developer in building an application that suits his needs. Highly recommended for people that feel free working in Python.\\
\textit{Stanford’s CoreNLP\\}
Provide set of tools written in Java for purpose of natural language processing. Initially was built to work only with English language, but latest releases support languages such as Arabic, Chinese, French, German and Spanish. It is an integrated framework easy to use for language manipulation on raw inputted text. The result it gives after initial analysis is a good starting point for building application with domain-specific problems. Besides low level natural language processing, it contains as well some traces of deep learning algorithms.\\
\textit{Text-Processing\\}
The text-processing is an open source API which returns simple JSON over HTTP web service for text mining and natural language processing. It is an API which supports speech tagging, chunking, sentiment analysis, phrase extraction and named entity recognition. As an open source solution it has its limitations, such as 1000 calls per day per IP. 
To get the sentiment of a text, users should do an HTTP request with form encoded data containing text to analyze. As a response, users will receive a JSON object with a label marking the sentiment (can be pos as positive, neg as negative or neutral) and a probability for each label.



