\newcommand*{\APIsOverviewPath}{04-framework/02-implementation/02-apis}
\newgeometry{bottom=1cm}

\subsection{Sentiment analysis APIs\label{sec:apis}}
Four different APIs were chosen to be tried and  evaluated against each other. 
They were chosen on the grounds of them being free and easily programmatically accessible regardless of the programming language used.
Thus making it feasible for us to invoke them from our framework.
Each of the APIs is called by making either a HTTP or HTTPS post request with a specific payload. The responses vary but all are parsed and stored in the format like the one shown in Listing \ref{lst:default-sentiment-json}.

\subsubsection*{Vivekn API}

\begin{description}
\singlespacing
 \item[Author:] Vivek Narayanan
 \item[Web url:] http://sentiment.vivekn.com/docs/api/
 \item[Database columns:] \inlinecode{sentiment\_api1} and \inlinecode{sentiment\_api1\_en}
\end{description}
As described on the API's website, the tool works by examining individual words and short sequences of words which then compares against a probability model. The probability model was built on a prelabeled test set of IMDb movie reviews
and it is based on the \emph{Fast and accurate sentiment classification using an enhanced Naive Bayes model} study \cite{DBLP:journals/corr/abs-1305-6143}.
The API is invoked via an HTTP post request with a \inlinecode{txt} payload that supplies comment's content. 
The request, response and the form in which the response was stored in \inlinecode{sentiment\_db} is presented in Table \ref{tab:vivekn-api}.

\newsavebox\vivekresponse
\newsavebox\viveknrequest
\newsavebox\vivekndb

\begin{lrbox}{\viveknrequest}
\begin{lstlisting}[style=json]
{  
  "txt": "Beautiful!" 
}
\end{lstlisting}
\end{lrbox}


\begin{lrbox}{\vivekresponse}
\begin{lstlisting}[style=json]
{ 
  "result": { 
    "sentiment": "Positive", 
    "confidence" : 72.422451 
  } 
}
\end{lstlisting}
\end{lrbox}

\begin{lrbox}{\vivekndb}
\begin{lstlisting}[style=json]
{
  "sentiment_label": "positive",
  "sentiment_stats": {
      "positive": 0.724,
      "negative": 0,
      "neutral" : 0
  }
}
\end{lstlisting}
\end{lrbox}

\begin{table}[H]
\centering
\singlespacing

\begin{tabularx}{0.9\textwidth}{ m{30mm}  X }
\hline  
  \textbf{request}  & \usebox\viveknrequest \\ \hline  
  \textbf{response} & \usebox\vivekresponse  \\ \hline  
  \textbf{stored} & \usebox\vivekndb  \\ \hline  

\end{tabularx}
\caption{Request and response payload to Vivekn API}
\label{tab:vivekn-api}

\end{table}
\restoregeometry


\subsubsection*{Text-processing API}
\begin{description}
\singlespacing
 \item[Web url:] http://text-processing.com/docs/sentiment.html
 \item[Database columns:] \inlinecode{sentiment\_api2} and \inlinecode{sentiment\_api2\_en}
\end{description}
This API is also trained on movie reviews but uses two binary classifiers. More specifically, it uses hierarchical classification by combining a subjectivity classifier and a polarity classifier. First, the subjectivity classifier determines whether the text is objective or subjective. If the text is objective, then it is classified as neutral. Otherwise the polarity classifier is used to determine if the text is positive or negative.  More details on the implementation can be found in articles  \cite{NaiveBayes}, \cite{Stopwords}, \cite{LowInformationFeatures} and \cite{HierarchicalClassification}.
However, the two classifier model introduces a caveat. Sentiment probabilities of positive and negative labels always sum up to one, while the neutral label is standalone and dominates the result if its probability is greater than 0.5.
The response, request and the stored JSON strings are shown in Table \ref{tab:text-processing-api}.
\newsavebox\tpresponse
\newsavebox\tprequest
\newsavebox\tpdb

\begin{lrbox}{\tprequest}
\begin{lstlisting}[style=json]
{  
  "text": "Beautiful!" 
}
\end{lstlisting}
\end{lrbox}


\begin{lrbox}{\tpresponse}
\begin{lstlisting}[style=json]
{
  "label": "neg",
  "probability": {
    "neutral": 0.38637609994709854,
    "neg": 0.68846305481785608,
    "pos": 0.31153694518214375
  }
}
\end{lstlisting}
\end{lrbox}

\begin{lrbox}{\tpdb}
\begin{lstlisting}[style=json]
{
  "sentiment_label": "negative",
  "sentiment_stats": {
      "positive": 0.311,
      "negative": 0.688,
      "neutral" : 0.386
  }
}
\end{lstlisting}
\end{lrbox}

\begin{table}[H]
\centering
\singlespacing

\begin{tabularx}{0.9\textwidth}{ m{30mm}  X }
\hline  
  \textbf{request}  & \usebox\tprequest \\ \hline  
  \textbf{response} & \usebox\tpresponse  \\ \hline  
  \textbf{stored} & \usebox\tpdb  \\ \hline  

\end{tabularx}
\caption{Request and response payload to Text-processing API}
\label{tab:text-processing-api}

\end{table}

\newpage

\subsubsection*{Indico API}
\begin{description}
\singlespacing
 \item[Web url:] https://indico.io/docs\#sentiment 
 \item[Web url (hq):] https://indico.io/docs\#sentiment\_hq 
 \item[Database columns:] \inlinecode{sentiment\_api3} and \inlinecode{sentiment\_api4 (hq)}
\end{description}
Even though we used all APIs as black boxes, the Indico API was treated as such even more so because implementation details weren't readily available on its website. 
It was also the only API that required registration in order to obtain an authorization key to access it. 
It did, however, provide two different endpoints, a regular sentiment analysis and a high quality (hq) version.
API's response is a number between 0 and 1. 
This number represents the likelihood that the analyzed text is positive. 
Meaning, values greater than 0.5 indicate positive sentiment, while values below that threshold indicate a negative sentiment. 
Since we were operating within a three label domain, a script was made to label all likelihoods $\in[a,b]$ as neutral to see how the accuracy (and other metrics) would change.
The response, request and the stored JSON are listed in Table \ref{tab:indico-api}.

\newsavebox\indicoresponse
\newsavebox\indicorequest
\newsavebox\indicodb

\begin{lrbox}{\indicorequest}
\begin{lstlisting}[style=json]
{  
  "text": "Beautiful!" 
}
\end{lstlisting}
\end{lrbox}


\begin{lrbox}{\indicoresponse}
\begin{lstlisting}[
style=json,
label={lst:indico-json}]
{
  "results": 0.3468102081511113
}
\end{lstlisting}
\end{lrbox}

\begin{lrbox}{\indicodb}
\begin{lstlisting}[style=json]
{
  "sentiment_label": "positive",
  "sentiment_stats": {
      "positive": 0.654,
      "negative": 0.346,
      "neutral" : 0
  }
}
\end{lstlisting}
\end{lrbox}

\begin{table}[H]
\centering
\singlespacing

\begin{tabularx}{0.9\textwidth}{ m{30mm}  X }
\hline  
  \textbf{request}  & \usebox\indicorequest \\ \hline  
  \textbf{response} & \usebox\indicoresponse  \\ \hline  
  \textbf{stored} & \usebox\indicodb  \\ \hline  

\end{tabularx}
\caption{Request and response payload to Indico and IndicoHq API endpoints}
\label{tab:indico-api}

\end{table}
