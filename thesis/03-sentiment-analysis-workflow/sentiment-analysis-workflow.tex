This chapter describes the workflow used to analyze the sentiment of social media comments and their corresponding posts.
In order to outline the workflow, a top down approach was taken where each subsequent section provides an ever more detailed insight into a particular step of the workflow.
The big picture is shown in Figure \ref{fig:analysis-workflow} and consists of four parts:
\begin{enumerate}
  \item Obtaining data
  \item Sentiment prediction using an API
  \item Determining real sentiment of data
  \item Evaluation of that API's performance
\end{enumerate}

First part is the simplest one and as such doesn't merit a more detailed recounting other than mentioning that we were provided with a small sample dataset which, most relevantly, contained about 6000 comments.

In the sections that follow, each of the three remaining parts are broken down into conceptual steps describing the methodology used whilst not cluttering it with too many implementation details.
Additionally, it is interesting to note that the first and third steps are done only once.
This means that, for each new API we want to use, the workflow for sentiment analysis effectively consists of only steps 2 and 4, namely sentiment prediction and performance evaluation.




The workflow  consists of two distinct parts: prediction and evaluation portion  blabla
how to evaluate? human input! 
...

...

...

...

See Figure \ref{fig:analysis-workflow}



The workflow  consists of two distinct parts: prediction and evaluation portion  blabla
how to evaluate? human input! 
...

...

...

...

See Figure \ref{fig:analysis-workflow}



The workflow  consists of two distinct parts: prediction and evaluation portion  blabla
how to evaluate? human input! 
...

...

...

...

See Figure \ref{fig:analysis-workflow}

\input{03-sentiment-analysis-workflow/diagrams/sentiment-analysis-workflow.tex}


% --- SECTION PREDICTON WORKFLOW ---
\section{Prediction workflow\label{sec:sentiment-prediction-workflow}}

Figure \ref{fig:prediction-workflow} shows the main concepts that build up the workflow of our sentiment analysis. 
Since the term \textit{workflow} can be a bit ambiguous, let us clarify exactly what we mean by it. In our case it is simply a python script named named \textit{automated\_sentiment\_analysis.py} that can be run manually, or scheduled to run on a server at desired times/intervals. 
Sections that follow will explain each step in more detail and will also provide motivation for some, perhaps not so obvious, choices.

\input{03-sentiment-analysis-workflow/diagrams/sentiment-prediction-workflow.tex}

\subsection*{Find new comments\label{sec:find-new-comments}}

This section is probably the most straight forward. Once run, the script scans the database looking for comments that don't have a sentiment record and inserts one. 
The inserted rows' sentiment columns default to a json shown in Listing \ref{lst:default-sentiment-json}. The reason for this particular choice of json and for using the json format in the first place is discussed at length in Section \ref{sec:design}.

\begin{lstlisting}[
style=json,
captionpos=b,
xleftmargin=.3\textwidth,
caption={Default sentiment json},
label={lst:default-sentiment-json}]
{
  "sentiment_label": "",
  "sentiment_stats": {
      "positive": 0,
      "negative": 0
      "neutral" : 0
  }
}
\end{lstlisting}
% --- SECTION TRANSLATE COMMENTS ---
\subsection*{Translate comments \label{sec:translate-comments}}
        Translate comments,{}
        Mark comments containing emojis, 
        Predict sentiment,
		Account for emojis, 
        Calculate aggregate sentiment for posts
by new we mean un analyzed


% --- SECTION EVALUATION WORKFLOW ---

\section{Evaluation workflow\label{sec:sentiment-evaluation-workflow}}

real sentiment input can be done either by hand orby the REST API GUI or the REST API curl calls

...

\input{03-sentiment-analysis-workflow/diagrams/sentiment-evaluation-workflow.tex}










% --- SECTION PREDICTON WORKFLOW ---
\section{Prediction workflow\label{sec:sentiment-prediction-workflow}}

Figure \ref{fig:prediction-workflow} shows the main concepts that build up the workflow of our sentiment analysis. 
Since the term \textit{workflow} can be a bit ambiguous, let us clarify exactly what we mean by it. In our case it is simply a python script named named \textit{automated\_sentiment\_analysis.py} that can be run manually, or scheduled to run on a server at desired times/intervals. 
Sections that follow will explain each step in more detail and will also provide motivation for some, perhaps not so obvious, choices.

\usesmartdiagramlibrary{additions}

\tikzset{
every shadow/.style={
fill=none,
shadow xshift=0pt,
shadow yshift=0pt}
}
\begin{figure}[ht]
	\vspace{3.5cm}
	\hspace{-3cm}
	\centering
	\smartdiagramset{
		text width=2.05cm,
		font=\scriptsize,
		uniform arrow color=true,
		uniform color list=white for 5 items,
		circular final arrow disabled=true, 
		circular distance=3.8cm,
		module shape=circle,
		border color=black,
		arrow tip=to,
		arrow color=black,
		arrow line width=1pt,
		additions={
			additional item font=\scriptsize,
			additional arrow line width=1pt,
			additional arrow color=black,
			additional item border color=black,
			additional item offset=1cm,
		}
	}
	\smartdiagramadd[circular diagram]{
		Find new comments,
		Translate comments,
		Predict sentiment,
		Adjust prediction to account for emojis, 
		Aggregate posts' sentiment
	}{ 
		above of module1/Start,
		right of module5/End
	}
	\smartdiagramconnect{to-}{module1/additional-module1}
	\smartdiagramconnect{-to}{module5/additional-module2}
	\vspace{0.5cm}

  \caption{Sentiment prediction workflow}

\label{fig:prediction-workflow}
\end{figure}


\subsection*{Find new comments\label{sec:find-new-comments}}

This section is probably the most straight forward. Once run, the script scans the database looking for comments that don't have a sentiment record and inserts one. 
The inserted rows' sentiment columns default to a json shown in Listing \ref{lst:default-sentiment-json}. The reason for this particular choice of json and for using the json format in the first place is discussed at length in Section \ref{sec:design}.

\begin{lstlisting}[
style=json,
captionpos=b,
xleftmargin=.3\textwidth,
caption={Default sentiment json},
label={lst:default-sentiment-json}]
{
  "sentiment_label": "",
  "sentiment_stats": {
      "positive": 0,
      "negative": 0
      "neutral" : 0
  }
}
\end{lstlisting}
% --- SECTION TRANSLATE COMMENTS ---
\subsection*{Translate comments \label{sec:translate-comments}}
        Translate comments,{}
        Mark comments containing emojis, 
        Predict sentiment,
		Account for emojis, 
        Calculate aggregate sentiment for posts
by new we mean un analyzed


% --- SECTION EVALUATION WORKFLOW ---

\section{Evaluation workflow\label{sec:sentiment-evaluation-workflow}}

real sentiment input can be done either by hand orby the REST API GUI or the REST API curl calls

...

\usesmartdiagramlibrary{additions}

\tikzset{
every shadow/.style={
fill=none,
shadow xshift=0pt,
shadow yshift=0pt}
}

\begin{figure}[ht]
	\vspace{3cm}
	\hspace{3cm}
	\centering
	\smartdiagramset{
		text width=2.05cm,
		font=\scriptsize,
		uniform arrow color=true,
		uniform color list=white for 1 items,
		module x sep=3.55cm,
		module shape=circle,
		border color=black,
		arrow tip=to,
		arrow color=black,
		arrow line width=1pt,
		back arrow disabled=true,
		additions={
			additional item font=\scriptsize,
			additional arrow line width=1pt,
			additional arrow color=black,
			additional item border color=black,
			additional item offset=1cm,
		}
	}
	\smartdiagramadd[flow diagram:horizontal]{
		Evaluation
	}{
		above left of module1/Real sentiment,
		left of module1/Predicted sentiment
	}
	\smartdiagramconnect{to-}{module1/additional-module1}
	\smartdiagramconnect{to-}{module1/additional-module2}
	\vspace{1cm}

  \caption{Sentiment evaluation workflow}

\label{fig:sentiment-evaluation-workflow}
\end{figure}










% --- SECTION PREDICTON WORKFLOW ---
\section{Prediction workflow\label{sec:sentiment-prediction-workflow}}

Figure \ref{fig:prediction-workflow} shows the main concepts that build up the workflow of our sentiment analysis. 
Since the term \textit{workflow} can be a bit ambiguous, let us clarify exactly what we mean by it. In our case it is simply a python script named named \textit{automated\_sentiment\_analysis.py} that can be run manually, or scheduled to run on a server at desired times/intervals. 
Sections that follow will explain each step in more detail and will also provide motivation for some, perhaps not so obvious, choices.

\usesmartdiagramlibrary{additions}

\tikzset{
every shadow/.style={
fill=none,
shadow xshift=0pt,
shadow yshift=0pt}
}
\begin{figure}[ht]
	\vspace{3.5cm}
	\hspace{-3cm}
	\centering
	\smartdiagramset{
		text width=2.05cm,
		font=\scriptsize,
		uniform arrow color=true,
		uniform color list=white for 5 items,
		circular final arrow disabled=true, 
		circular distance=3.8cm,
		module shape=circle,
		border color=black,
		arrow tip=to,
		arrow color=black,
		arrow line width=1pt,
		additions={
			additional item font=\scriptsize,
			additional arrow line width=1pt,
			additional arrow color=black,
			additional item border color=black,
			additional item offset=1cm,
		}
	}
	\smartdiagramadd[circular diagram]{
		Find new comments,
		Translate comments,
		Predict sentiment,
		Adjust prediction to account for emojis, 
		Aggregate posts' sentiment
	}{ 
		above of module1/Start,
		right of module5/End
	}
	\smartdiagramconnect{to-}{module1/additional-module1}
	\smartdiagramconnect{-to}{module5/additional-module2}
	\vspace{0.5cm}

  \caption{Sentiment prediction workflow}

\label{fig:prediction-workflow}
\end{figure}


\subsection*{Find new comments\label{sec:find-new-comments}}

This section is probably the most straight forward. Once run, the script scans the database looking for comments that don't have a sentiment record and inserts one. 
The inserted rows' sentiment columns default to a json shown in Listing \ref{lst:default-sentiment-json}. The reason for this particular choice of json and for using the json format in the first place is discussed at length in Section \ref{sec:design}.

\begin{lstlisting}[
style=json,
captionpos=b,
xleftmargin=.3\textwidth,
caption={Default sentiment json},
label={lst:default-sentiment-json}]
{
  "sentiment_label": "",
  "sentiment_stats": {
      "positive": 0,
      "negative": 0
      "neutral" : 0
  }
}
\end{lstlisting}
% --- SECTION TRANSLATE COMMENTS ---
\subsection*{Translate comments \label{sec:translate-comments}}
        Translate comments,{}
        Mark comments containing emojis, 
        Predict sentiment,
		Account for emojis, 
        Calculate aggregate sentiment for posts
by new we mean un analyzed


% --- SECTION EVALUATION WORKFLOW ---

\section{Evaluation workflow\label{sec:sentiment-evaluation-workflow}}

real sentiment input can be done either by hand orby the REST API GUI or the REST API curl calls

...

\usesmartdiagramlibrary{additions}

\tikzset{
every shadow/.style={
fill=none,
shadow xshift=0pt,
shadow yshift=0pt}
}

\begin{figure}[ht]
	\vspace{3cm}
	\hspace{3cm}
	\centering
	\smartdiagramset{
		text width=2.05cm,
		font=\scriptsize,
		uniform arrow color=true,
		uniform color list=white for 1 items,
		module x sep=3.55cm,
		module shape=circle,
		border color=black,
		arrow tip=to,
		arrow color=black,
		arrow line width=1pt,
		back arrow disabled=true,
		additions={
			additional item font=\scriptsize,
			additional arrow line width=1pt,
			additional arrow color=black,
			additional item border color=black,
			additional item offset=1cm,
		}
	}
	\smartdiagramadd[flow diagram:horizontal]{
		Evaluation
	}{
		above left of module1/Real sentiment,
		left of module1/Predicted sentiment
	}
	\smartdiagramconnect{to-}{module1/additional-module1}
	\smartdiagramconnect{to-}{module1/additional-module2}
	\vspace{1cm}

  \caption{Sentiment evaluation workflow}

\label{fig:sentiment-evaluation-workflow}
\end{figure}










% --- SECTION PREDICTON WORKFLOW ---
\section{Sentiment prediction workflow\label{sec:sentiment-prediction-workflow}}
Let's assume we have access to an API for sentiment prediction. And by having access we mean being able to programmatically call the API with a text payload and have it return a prediction in some data format. The end goal is to analyze sentiment of all the comments in our sample dataset and aggregate the obtained data on a per post basis in order to infer whether it is was positively or negatively received, or even if it had no emotional impact whatsoever. And we want this to be done automatically, practically with a push of a proverbial button. By automatizing the process, it is easy to see how it can derive value for possible future ventures that extend far beyond our modest 6000 comment database.

\usesmartdiagramlibrary{additions}

\tikzset{
every shadow/.style={
fill=none,
shadow xshift=0pt,
shadow yshift=0pt}
}
\begin{figure}[ht]
	\vspace{3.5cm}
	\hspace{-3cm}
	\centering
	\smartdiagramset{
		text width=2.05cm,
		font=\scriptsize,
		uniform arrow color=true,
		uniform color list=white for 5 items,
		circular final arrow disabled=true, 
		circular distance=3.8cm,
		module shape=circle,
		border color=black,
		arrow tip=to,
		arrow color=black,
		arrow line width=1pt,
		additions={
			additional item font=\scriptsize,
			additional arrow line width=1pt,
			additional arrow color=black,
			additional item border color=black,
			additional item offset=1cm,
		}
	}
	\smartdiagramadd[circular diagram]{
		Find new comments,
		Translate comments,
		Predict sentiment,
		Adjust prediction to account for emojis, 
		Aggregate posts' sentiment
	}{ 
		above of module1/Start,
		right of module5/End
	}
	\smartdiagramconnect{to-}{module1/additional-module1}
	\smartdiagramconnect{-to}{module5/additional-module2}
	\vspace{0.5cm}

  \caption{Sentiment prediction workflow}

\label{fig:prediction-workflow}
\end{figure}


Figure \ref{fig:prediction-workflow} shows the main concepts that build up the workflow of our sentiment analysis. 
Since the term \textit{workflow} can be a bit ambiguous, let us clarify exactly what we mean by it. In our case it is simply a python script named named \textit{automated\_sentiment\_analysis.py} that can be run manually, or scheduled to run on a server at desired times/intervals. 
Sections that follow will explain each step in more detail and will also provide motivation for some, perhaps not so obvious, choices.


\subsection*{Find new comments\label{sec:find-new-comments}}
This part  quite straight forward 
 Once run, the script scans the database looking for comments that don't have a sentiment record and inserts one.  

 for the sake of completeness
The inserted rows' sentiment columns default to a json shown in Listing \ref{lst:default-sentiment-json}. The reason for this particular choice of json and for using the json format in the first place is discussed at length in Section \ref{sec:design}.

\begin{lstlisting}[
style=json,
captionpos=b,
xleftmargin=.3\textwidth,
caption={Default sentiment json},
label={lst:default-sentiment-json}]
{
  "sentiment_label": "",
  "sentiment_stats": {
      "positive": 0,
      "negative": 0
      "neutral" : 0
  }
}
\end{lstlisting}
% --- SECTION TRANSLATE COMMENTS ---
\subsection*{Translate comments \label{sec:translate-comments}}
        Translate comments,{}
        Mark comments containing emojis, 
        Predict sentiment,
		Account for emojis, 
        Calculate aggregate sentiment for posts
by new we mean unanalyzed

...

% --- REAL SENTIMENT  WORKFLOW ---

\section{Determining real sentiment workflow\label{sec:determining-real-sentiment-workflow}}
How do we know our predictions are any good?
real sentiment input can be done either by hand orby the REST API GUI or the REST API curl calls


\input{03-sentiment-analysis-workflow/diagrams/real-sentiment-workflow.tex}

% --- SECTION EVALUATION WORKFLOW ---

\section{Evaluation workflow\label{sec:sentiment-evaluation-workflow}}


performance evaluation of that particular API.
how to evaluate? human input! 


\usesmartdiagramlibrary{additions}

\tikzset{
every shadow/.style={
fill=none,
shadow xshift=0pt,
shadow yshift=0pt}
}

\begin{figure}[ht]
	\vspace{3cm}
	\hspace{3cm}
	\centering
	\smartdiagramset{
		text width=2.05cm,
		font=\scriptsize,
		uniform arrow color=true,
		uniform color list=white for 1 items,
		module x sep=3.55cm,
		module shape=circle,
		border color=black,
		arrow tip=to,
		arrow color=black,
		arrow line width=1pt,
		back arrow disabled=true,
		additions={
			additional item font=\scriptsize,
			additional arrow line width=1pt,
			additional arrow color=black,
			additional item border color=black,
			additional item offset=1cm,
		}
	}
	\smartdiagramadd[flow diagram:horizontal]{
		Evaluation
	}{
		above left of module1/Real sentiment,
		left of module1/Predicted sentiment
	}
	\smartdiagramconnect{to-}{module1/additional-module1}
	\smartdiagramconnect{to-}{module1/additional-module2}
	\vspace{1cm}

  \caption{Sentiment evaluation workflow}

\label{fig:sentiment-evaluation-workflow}
\end{figure}

