

The workflow  consists of two distinct parts: prediction and evaluation portion  blabla
how to evaluate? human input! 
...

...

...

...

See Figure \ref{fig:analysis-workflow}



The workflow  consists of two distinct parts: prediction and evaluation portion  blabla
how to evaluate? human input! 
...

...

...

...

See Figure \ref{fig:analysis-workflow}



The workflow  consists of two distinct parts: prediction and evaluation portion  blabla
how to evaluate? human input! 
...

...

...

...

See Figure \ref{fig:analysis-workflow}



The workflow  consists of two distinct parts: prediction and evaluation portion  blabla
how to evaluate? human input! 
...

...

...

...

See Figure \ref{fig:analysis-workflow}

\input{03-sentiment-analysis-workflow/diagrams/sentiment-analysis-workflow.tex}


% --- SECTION PREDICTON WORKFLOW ---
\section{Prediction workflow\label{sec:sentiment-prediction-workflow}}

Figure \ref{fig:prediction-workflow} shows the main concepts that build up the workflow of our sentiment analysis. 
Since the term \textit{workflow} can be a bit ambiguous, let us clarify exactly what we mean by it. In our case it is simply a python script named named \textit{automated\_sentiment\_analysis.py} that can be run manually, or scheduled to run on a server at desired times/intervals. 
Sections that follow will explain each step in more detail and will also provide motivation for some, perhaps not so obvious, choices.

\input{03-sentiment-analysis-workflow/diagrams/sentiment-prediction-workflow.tex}

\subsection*{Find new comments\label{sec:find-new-comments}}

This section is probably the most straight forward. Once run, the script scans the database looking for comments that don't have a sentiment record and inserts one. 
The inserted rows' sentiment columns default to a json shown in Listing \ref{lst:default-sentiment-json}. The reason for this particular choice of json and for using the json format in the first place is discussed at length in Section \ref{sec:design}.

\begin{lstlisting}[
style=json,
captionpos=b,
xleftmargin=.3\textwidth,
caption={Default sentiment json},
label={lst:default-sentiment-json}]
{
  "sentiment_label": "",
  "sentiment_stats": {
      "positive": 0,
      "negative": 0
      "neutral" : 0
  }
}
\end{lstlisting}
% --- SECTION TRANSLATE COMMENTS ---
\subsection*{Translate comments \label{sec:translate-comments}}
        Translate comments,{}
        Mark comments containing emojis, 
        Predict sentiment,
		Account for emojis, 
        Calculate aggregate sentiment for posts
by new we mean un analyzed


% --- SECTION EVALUATION WORKFLOW ---

\section{Evaluation workflow\label{sec:sentiment-evaluation-workflow}}

real sentiment input can be done either by hand orby the REST API GUI or the REST API curl calls

...

\input{03-sentiment-analysis-workflow/diagrams/sentiment-evaluation-workflow.tex}










% --- SECTION PREDICTON WORKFLOW ---
\section{Prediction workflow\label{sec:sentiment-prediction-workflow}}

Figure \ref{fig:prediction-workflow} shows the main concepts that build up the workflow of our sentiment analysis. 
Since the term \textit{workflow} can be a bit ambiguous, let us clarify exactly what we mean by it. In our case it is simply a python script named named \textit{automated\_sentiment\_analysis.py} that can be run manually, or scheduled to run on a server at desired times/intervals. 
Sections that follow will explain each step in more detail and will also provide motivation for some, perhaps not so obvious, choices.

\usesmartdiagramlibrary{additions}

\tikzset{
every shadow/.style={
fill=none,
shadow xshift=0pt,
shadow yshift=0pt}
}
\begin{figure}[ht]
	\vspace{3.5cm}
	\hspace{-3cm}
	\centering
	\smartdiagramset{
		text width=2.05cm,
		font=\scriptsize,
		uniform arrow color=true,
		uniform color list=white for 5 items,
		circular final arrow disabled=true, 
		circular distance=3.8cm,
		module shape=circle,
		border color=black,
		arrow tip=to,
		arrow color=black,
		arrow line width=1pt,
		additions={
			additional item font=\scriptsize,
			additional arrow line width=1pt,
			additional arrow color=black,
			additional item border color=black,
			additional item offset=1cm,
		}
	}
	\smartdiagramadd[circular diagram]{
		Find new comments,
		Translate comments,
		Predict sentiment,
		Adjust prediction to account for emojis, 
		Aggregate posts' sentiment
	}{ 
		above of module1/Start,
		right of module5/End
	}
	\smartdiagramconnect{to-}{module1/additional-module1}
	\smartdiagramconnect{-to}{module5/additional-module2}
	\vspace{0.5cm}

  \caption{Sentiment prediction workflow}

\label{fig:prediction-workflow}
\end{figure}


\subsection*{Find new comments\label{sec:find-new-comments}}

This section is probably the most straight forward. Once run, the script scans the database looking for comments that don't have a sentiment record and inserts one. 
The inserted rows' sentiment columns default to a json shown in Listing \ref{lst:default-sentiment-json}. The reason for this particular choice of json and for using the json format in the first place is discussed at length in Section \ref{sec:design}.

\begin{lstlisting}[
style=json,
captionpos=b,
xleftmargin=.3\textwidth,
caption={Default sentiment json},
label={lst:default-sentiment-json}]
{
  "sentiment_label": "",
  "sentiment_stats": {
      "positive": 0,
      "negative": 0
      "neutral" : 0
  }
}
\end{lstlisting}
% --- SECTION TRANSLATE COMMENTS ---
\subsection*{Translate comments \label{sec:translate-comments}}
        Translate comments,{}
        Mark comments containing emojis, 
        Predict sentiment,
		Account for emojis, 
        Calculate aggregate sentiment for posts
by new we mean un analyzed


% --- SECTION EVALUATION WORKFLOW ---

\section{Evaluation workflow\label{sec:sentiment-evaluation-workflow}}

real sentiment input can be done either by hand orby the REST API GUI or the REST API curl calls

...

\usesmartdiagramlibrary{additions}

\tikzset{
every shadow/.style={
fill=none,
shadow xshift=0pt,
shadow yshift=0pt}
}

\begin{figure}[ht]
	\vspace{3cm}
	\hspace{3cm}
	\centering
	\smartdiagramset{
		text width=2.05cm,
		font=\scriptsize,
		uniform arrow color=true,
		uniform color list=white for 1 items,
		module x sep=3.55cm,
		module shape=circle,
		border color=black,
		arrow tip=to,
		arrow color=black,
		arrow line width=1pt,
		back arrow disabled=true,
		additions={
			additional item font=\scriptsize,
			additional arrow line width=1pt,
			additional arrow color=black,
			additional item border color=black,
			additional item offset=1cm,
		}
	}
	\smartdiagramadd[flow diagram:horizontal]{
		Evaluation
	}{
		above left of module1/Real sentiment,
		left of module1/Predicted sentiment
	}
	\smartdiagramconnect{to-}{module1/additional-module1}
	\smartdiagramconnect{to-}{module1/additional-module2}
	\vspace{1cm}

  \caption{Sentiment evaluation workflow}

\label{fig:sentiment-evaluation-workflow}
\end{figure}










% --- SECTION PREDICTON WORKFLOW ---
\section{Prediction workflow\label{sec:sentiment-prediction-workflow}}

Figure \ref{fig:prediction-workflow} shows the main concepts that build up the workflow of our sentiment analysis. 
Since the term \textit{workflow} can be a bit ambiguous, let us clarify exactly what we mean by it. In our case it is simply a python script named named \textit{automated\_sentiment\_analysis.py} that can be run manually, or scheduled to run on a server at desired times/intervals. 
Sections that follow will explain each step in more detail and will also provide motivation for some, perhaps not so obvious, choices.

\usesmartdiagramlibrary{additions}

\tikzset{
every shadow/.style={
fill=none,
shadow xshift=0pt,
shadow yshift=0pt}
}
\begin{figure}[ht]
	\vspace{3.5cm}
	\hspace{-3cm}
	\centering
	\smartdiagramset{
		text width=2.05cm,
		font=\scriptsize,
		uniform arrow color=true,
		uniform color list=white for 5 items,
		circular final arrow disabled=true, 
		circular distance=3.8cm,
		module shape=circle,
		border color=black,
		arrow tip=to,
		arrow color=black,
		arrow line width=1pt,
		additions={
			additional item font=\scriptsize,
			additional arrow line width=1pt,
			additional arrow color=black,
			additional item border color=black,
			additional item offset=1cm,
		}
	}
	\smartdiagramadd[circular diagram]{
		Find new comments,
		Translate comments,
		Predict sentiment,
		Adjust prediction to account for emojis, 
		Aggregate posts' sentiment
	}{ 
		above of module1/Start,
		right of module5/End
	}
	\smartdiagramconnect{to-}{module1/additional-module1}
	\smartdiagramconnect{-to}{module5/additional-module2}
	\vspace{0.5cm}

  \caption{Sentiment prediction workflow}

\label{fig:prediction-workflow}
\end{figure}


\subsection*{Find new comments\label{sec:find-new-comments}}

This section is probably the most straight forward. Once run, the script scans the database looking for comments that don't have a sentiment record and inserts one. 
The inserted rows' sentiment columns default to a json shown in Listing \ref{lst:default-sentiment-json}. The reason for this particular choice of json and for using the json format in the first place is discussed at length in Section \ref{sec:design}.

\begin{lstlisting}[
style=json,
captionpos=b,
xleftmargin=.3\textwidth,
caption={Default sentiment json},
label={lst:default-sentiment-json}]
{
  "sentiment_label": "",
  "sentiment_stats": {
      "positive": 0,
      "negative": 0
      "neutral" : 0
  }
}
\end{lstlisting}
% --- SECTION TRANSLATE COMMENTS ---
\subsection*{Translate comments \label{sec:translate-comments}}
        Translate comments,{}
        Mark comments containing emojis, 
        Predict sentiment,
		Account for emojis, 
        Calculate aggregate sentiment for posts
by new we mean un analyzed


% --- SECTION EVALUATION WORKFLOW ---

\section{Evaluation workflow\label{sec:sentiment-evaluation-workflow}}

real sentiment input can be done either by hand orby the REST API GUI or the REST API curl calls

...

\usesmartdiagramlibrary{additions}

\tikzset{
every shadow/.style={
fill=none,
shadow xshift=0pt,
shadow yshift=0pt}
}

\begin{figure}[ht]
	\vspace{3cm}
	\hspace{3cm}
	\centering
	\smartdiagramset{
		text width=2.05cm,
		font=\scriptsize,
		uniform arrow color=true,
		uniform color list=white for 1 items,
		module x sep=3.55cm,
		module shape=circle,
		border color=black,
		arrow tip=to,
		arrow color=black,
		arrow line width=1pt,
		back arrow disabled=true,
		additions={
			additional item font=\scriptsize,
			additional arrow line width=1pt,
			additional arrow color=black,
			additional item border color=black,
			additional item offset=1cm,
		}
	}
	\smartdiagramadd[flow diagram:horizontal]{
		Evaluation
	}{
		above left of module1/Real sentiment,
		left of module1/Predicted sentiment
	}
	\smartdiagramconnect{to-}{module1/additional-module1}
	\smartdiagramconnect{to-}{module1/additional-module2}
	\vspace{1cm}

  \caption{Sentiment evaluation workflow}

\label{fig:sentiment-evaluation-workflow}
\end{figure}










% --- SECTION PREDICTON WORKFLOW ---
\section{Prediction workflow\label{sec:sentiment-prediction-workflow}}

Figure \ref{fig:prediction-workflow} shows the main concepts that build up the workflow of our sentiment analysis. 
Since the term \textit{workflow} can be a bit ambiguous, let us clarify exactly what we mean by it. In our case it is simply a python script named named \textit{automated\_sentiment\_analysis.py} that can be run manually, or scheduled to run on a server at desired times/intervals. 
Sections that follow will explain each step in more detail and will also provide motivation for some, perhaps not so obvious, choices.

\usesmartdiagramlibrary{additions}

\tikzset{
every shadow/.style={
fill=none,
shadow xshift=0pt,
shadow yshift=0pt}
}
\begin{figure}[ht]
	\vspace{3.5cm}
	\hspace{-3cm}
	\centering
	\smartdiagramset{
		text width=2.05cm,
		font=\scriptsize,
		uniform arrow color=true,
		uniform color list=white for 5 items,
		circular final arrow disabled=true, 
		circular distance=3.8cm,
		module shape=circle,
		border color=black,
		arrow tip=to,
		arrow color=black,
		arrow line width=1pt,
		additions={
			additional item font=\scriptsize,
			additional arrow line width=1pt,
			additional arrow color=black,
			additional item border color=black,
			additional item offset=1cm,
		}
	}
	\smartdiagramadd[circular diagram]{
		Find new comments,
		Translate comments,
		Predict sentiment,
		Adjust prediction to account for emojis, 
		Aggregate posts' sentiment
	}{ 
		above of module1/Start,
		right of module5/End
	}
	\smartdiagramconnect{to-}{module1/additional-module1}
	\smartdiagramconnect{-to}{module5/additional-module2}
	\vspace{0.5cm}

  \caption{Sentiment prediction workflow}

\label{fig:prediction-workflow}
\end{figure}


\subsection*{Find new comments\label{sec:find-new-comments}}

This section is probably the most straight forward. Once run, the script scans the database looking for comments that don't have a sentiment record and inserts one. 
The inserted rows' sentiment columns default to a json shown in Listing \ref{lst:default-sentiment-json}. The reason for this particular choice of json and for using the json format in the first place is discussed at length in Section \ref{sec:design}.

\begin{lstlisting}[
style=json,
captionpos=b,
xleftmargin=.3\textwidth,
caption={Default sentiment json},
label={lst:default-sentiment-json}]
{
  "sentiment_label": "",
  "sentiment_stats": {
      "positive": 0,
      "negative": 0
      "neutral" : 0
  }
}
\end{lstlisting}
% --- SECTION TRANSLATE COMMENTS ---
\subsection*{Translate comments \label{sec:translate-comments}}
        Translate comments,{}
        Mark comments containing emojis, 
        Predict sentiment,
		Account for emojis, 
        Calculate aggregate sentiment for posts
by new we mean un analyzed


% --- SECTION EVALUATION WORKFLOW ---

\section{Evaluation workflow\label{sec:sentiment-evaluation-workflow}}

real sentiment input can be done either by hand orby the REST API GUI or the REST API curl calls

...

\usesmartdiagramlibrary{additions}

\tikzset{
every shadow/.style={
fill=none,
shadow xshift=0pt,
shadow yshift=0pt}
}

\begin{figure}[ht]
	\vspace{3cm}
	\hspace{3cm}
	\centering
	\smartdiagramset{
		text width=2.05cm,
		font=\scriptsize,
		uniform arrow color=true,
		uniform color list=white for 1 items,
		module x sep=3.55cm,
		module shape=circle,
		border color=black,
		arrow tip=to,
		arrow color=black,
		arrow line width=1pt,
		back arrow disabled=true,
		additions={
			additional item font=\scriptsize,
			additional arrow line width=1pt,
			additional arrow color=black,
			additional item border color=black,
			additional item offset=1cm,
		}
	}
	\smartdiagramadd[flow diagram:horizontal]{
		Evaluation
	}{
		above left of module1/Real sentiment,
		left of module1/Predicted sentiment
	}
	\smartdiagramconnect{to-}{module1/additional-module1}
	\smartdiagramconnect{to-}{module1/additional-module2}
	\vspace{1cm}

  \caption{Sentiment evaluation workflow}

\label{fig:sentiment-evaluation-workflow}
\end{figure}







